\documentclass{article}
\usepackage{tkz-euclide}
\title{Dreieck}
\setcounter{secnumdepth}{0}
\begin{document}
%%	 \maketitle
	 \begin{section}{Trigonometrie}
	 	\begin{tikzpicture}
	 		\newcommand{\scale}{2}
	 		%fixed points
	 		
	 		\tkzDefPoints{0/0/A,10/0/B,3.6/4.8/C}
	 		\tkzDefPointBy[projection=onto A--B ](C) \tkzGetPoint{qp}
	 		\tkzDefMidPoint(A,B)					 \tkzGetPoint{c}
	 		\tkzDefMidPoint(B,C)					 \tkzGetPoint{a}
	 		\tkzDefMidPoint(C,A)					 \tkzGetPoint{b}
	 		\tkzDefMidPoint(C,qp)					 \tkzGetPoint{h}
	 		
	 		% Drawing
	 		
	 		\tkzDrawPolygon(A,B,C)
	 		\tkzDrawSegments[dashed](C,qp)
	 		\tkzDrawPoints(A,B,C)
	 		%\tkzLabelPoints(A,B,qp)
	 		%\tkzLabelPoints[above](C)
	 		\tkzLabelPoints[above right,scale=\scale](a)
	 		\tkzLabelPoints[scale=\scale](b,c,h)
	 		\tkzMarkRightAngle(A,C,B)
	 		\draw[thick] pic["$\theta$"scale=\scale,draw, thick , <->, angle eccentricity=2/3, angle radius=3.0cm] {angle=C--B--A};
	 	\end{tikzpicture}
	 	\begin{subsection}{Aufgabenstellung}
	 		Schreiben Sie eine Anwendung welche zwei beliebige Angaben eines rechtwinkligen Dreiecks akzeptiert, und die restlichen Werte bestimmen kann.\\
	 		Die Möglichen Angaben sind:
	 		\begin{itemize}
	 			\item Die Ankathete $a$
	 			\item Die Gegenkathete $b$
	 			\item Die Hypotenuse $c$
	 			\item Die Höhe $h$
	 			\item Der Winkel $\theta$
	 		\end{itemize}
 			
	 	\end{subsection}
	 	\begin{subsection}{Erweiterungen}
	 		
	 	\end{subsection}
	\end{section}
\end{document}

