\documentclass{article}
\title{Dreieck}
\setcounter{secnumdepth}{0}
\begin{document}
%%	 \maketitle
	 \begin{section}{Trigonometrie}
	 	\begin{subsection}{Aufgabenstellung}
	 		Schreiben Sie eine Anwendung welche zwei beliebige Angaben eines rechtwinkligen Dreiecks akzeptiert, und die restlichen Werte bestimmen kann.\\
	 		Die Möglichen Angaben sind:
	 		\begin{itemize}
	 			\item Ankathete (an $\theta$)
	 			\item Gegenkathete (gegenüber $\theta$)
	 			\item $\theta$
	 			\item Hypotenuse
	 			\item Höhe (zur Hypotenuse)
	 		\end{itemize}
	 	\end{subsection}
	 \end{section}
\end{document}

